\documentclass{resume}
\usepackage{xcolor}
\usepackage{enumitem}

\author{https://aravindvasu.dev}{Aravind Vasudevan}
\email{avasude2@andrew.cmu.edu}
\phone{+14126087267}{(412) 608-7267}
\domain{https://aravindvasu.dev/about/}{https://aravindvasu.dev}
\github{github.com/AravindVasudev}{https://github.com/AravindVasudev}

\setlist[itemize]{noitemsep, topsep=0pt, leftmargin=*}

\begin{document}
\maketitle
\section*{Education}
\titlerule
\noindent
\\
\textbf{Carnegie Mellon University} \hfill \textit{Pittsburgh, PA} \\
{\small Master of Software Engineering -- Scalable Systems} \hfill \textit{\small December 2020}\\
\textbf{Anna University} \hfill \textit{Chennai, India} \\
{\small Bachelor of Engineering -- Computer Science} \hfill \textit{\small May 2018}

\section*{Coursework}
\titlerule
\noindent
\\
Methods: Deciding What to Design, Models of Software Systems, Managing Software Development, Introduction to Machine Learning, Design and Analysis of Algorithms, Object Oriented Analysis \& Design, Introduction to Computer Systems.
\section*{Experience}
\titlerule
\noindent
\\
\textbf{Zoho Corporation} \hfill \textit{Chennai, India} \\
{\small Member Technical Staff} \hfill \textit{\small June 2018 - May 2019}
\begin{itemize}
  \item Implemented an annotation-based transaction handling framework using Aspect-Oriented Programming.
  \item Designed an internal HTTP connection pool library to work with multiple services simultaneously.
  \item Developed a library that loads multiple i18n keysets without collision in a project split across multiple repositories.
  \item Built a tool to manage configuration validation and building across multiple codebases.
\end{itemize}
\textbf{Zoho Corporation} \hfill \textit{Chennai, India} \\
{\small Project Trainee} \hfill \textit{\small December 2017 - March 2018}
\begin{itemize}
  \item Architected an inter-service communication framework using observable pattern that can trigger distributed functions.
  \item Decoupled features in Zoho CRM into multiple repositories and designed them to work together as libraries.
  \item Rewrote several classes to accommodate to support migration to our custom architecture.
\end{itemize}
\textbf{Tata Consultancy Services} \hfill \textit{Chennai, India} \\
{\small Project Intern} \hfill \textit{\small June 2017 - July 2017}
\begin{itemize}
  \item Prototyped a chatbot to simplify data lookup in Planatics, an internal financial data management tool, using RASA NLU.
  \item Learned to build microservices and REST API endpoints for service interaction.
\end{itemize}
\section*{Personal Projects}
\titlerule
\noindent
\\
\textbf{Yabber} \hfill \textit{MEAN Stack}
\begin{itemize}
  \item Concurrent group chat application using MEAN Stack and Redis.
  \item Software engineering practices such as use case modeling, interface designing, and data modeling were followed.
  \item Software Requirement Specification, and Design and Data Modeling documentation were made before development.
  \item The schema was realized using a modified ER diagram for Mongo DB and Redis.
\end{itemize}
\textbf{Lendr} \hfill \textit{Spring Framework}
\begin{itemize}
  \item Hybrid mobile application using ionic that facilitates lending-borrowing of everyday commodities within a location.
  \item The backend of the application was built using Spring framework and MySQL.
\end{itemize}
\textbf{Clip-Sync} \hfill \textit{Flask framework}
\begin{itemize}
  \item Clip-Sync allows sharing a system's clipboard within a network and is useful when working on documents as a group.
  \item The server component was built using Flask and the real-time clipboard sharing is handled using Flask-SocketIO.
  \item The clipboard of the system is shared via HTTP.
\end{itemize}
\textbf{2048 AI} \hfill \textit{Expectimax Algorithm}
\begin{itemize}
  \item AI solver for the game 2048 that plays the game using the heuristic expectimax tree search algorithm.
  \item The 2048 game interface is implemented using Angular 6 and SCSS.
\end{itemize}
\textbf{Defect Predictor} \hfill \textit{Machine Learning}
\begin{itemize}
  \item A ML tool that predicts the numbers of bugs a project might occur in a project based on the team's previous experience.
  \item Tested the data with various regression models and finalized AdaBoost regressor based on test and validation error.
  \item Grid Search was used to tune the hyperparameters for the models.
\end{itemize}
\section*{Skills}
\titlerule
\noindent
\\
\textbf{Web Development} Node.js, Angular, Struts, Spring, JQuery, SCSS, Handlebars, PHP, Docker, HTML, CSS, Javascript. \\
\textbf{Database} MySQL, MongoDB, Redis. \\
\textbf{Programming Languages} Java, Python, C, C++, Shell Scripting, R.\\
\end{document}
